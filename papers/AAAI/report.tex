% File: report.tex

\documentclass[letterpaper]{article}
\usepackage{aaai}
\usepackage{times}
\usepackage{helvet}
\usepackage{courier}
\frenchspacing
\pdfinfo{
/Title (Primitive Recognition using Inverse Q-Table Matching)
/Subject (In Proceedings AAAI)
/Author (Bradley Hayes, Brian Scassellati, and Thaddeus Diamond)
}
  
\begin{document}
\title{Scene Decomposition using Inverse Q-Table Matching}
\author{Bradley Hayes, Brian Scassellati, and Thaddeus Diamond\\
Yale University\\
Arthur K. Watson Hall\\
51 Prospect St\\
New Haven, Connecticut 06511\\
}

\maketitle

\begin{abstract}
Several variations of Q-Learning have reliably demonstrated that reinforcement
learning is a viable technique for creating learned skills in robots.  However,
very little work has focused on using sensory data to generate an approximation
of which Q-Table most reliably approximates the skill just observed.  We present
herein a system which is able to consistently provide accurate decompositions of
a scene into simple, learned primitives.  The system is then able to take this
frame-by-frame decomposition to provide an approximation for a suitable 
hierarchical structure of the task being performed.  This structure is composed
entirely of either previously learned primitives or unknowns.  Once those
learned actions have been successfully mapped over time, the entire scene, or
merely portions of it, can be replicated by robotic agents in a novel
environment.  
\end{abstract}

\section{Introduction}
\label{sec:intro}
In this section we introduce the problem.

\section{Background}
\label{sec:background}
In this section we discuss relevant background to the problem.

\section{Primitive Recognition}
\label{sec:recognition}
In this section we discuss our work on primitive recognition.

\section{Results}
\label{sec:result}
In this section we discuss our experimental results.

\section{Related and Future Work}
\label{sec:future}
In this section we discuss future work (hint at SHL?)

\section{Conclusions}
\label{sec:conclusions}
In this section we conclude.

\section{Acknowledgements}
\label{sec:acknowledgements}
We would like to thank... grant numbers?  Brad's grad school funding?  God?
Wayne Brady?

\bibliography{report.bib}
\bibliographystyle{aaai}

\end{document}
