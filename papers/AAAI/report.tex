% File: report.tex

\documentclass[letterpaper]{article}
\usepackage{aaai}
\usepackage{times}
\usepackage{helvet}
\usepackage{courier}
\frenchspacing
\pdfoutput=1
\pdfinfo{
/Title (Using A Hierarchical Architecture for Collaborative Social Tasks to Reinforce and Recognize Action Primitives)
/Subject (In Proceedings AAAI)
/Author (Bradley Hayes, Thaddeus Diamond, and Brian Scassellati)
}
  
\begin{document}
\title{Using A Hierarchical Architecture for Collaborative Social Tasks to Reinforce and Recognize Action Primitives}
\author{Bradley Hayes, Thaddeus Diamond, and Brian Scassellati\\
Yale University\\
Arthur K. Watson Hall\\
51 Prospect St\\
New Haven, Connecticut 06511\\
}

\maketitle

\begin{abstract}
Primitive skill acquisition for robots has been accomplished through a variety
of techniques, many utilizing Q-Learning. Recent work has shown that with human
assistance, complex actions can be learned with limited training data. The
resulting skill representations can be leveraged to replay flexible behaviors
that dramatically increase the utility of the robot. Despite this, existing
systems have yet to address the greater problem of cooperative action execution.
For a robot to effectively cooperate with either human or robotic coworkers,
it is necessary that the robot be capable of modeling and recognizing its
coworker’s behaviors within its environment. We approach the task of
understanding and predicting coworker actions by providing a real-time method
allowing a robot to solve the inverse problem of skill execution: skill
recognition. We demonstrate our results through the recognition of selected
American Sign Language gestures. Our algorithm provides a coded timeline of
primitive actions generated in real-time, correcting its hypotheses as more
information is presented. This data can then be directly used in a multitude
of ways, from within a planning system for the coordination of a robot's
actions to building a skill hierarchy fully decomposing a complex cooperative 
task into primitive actions.
\end{abstract}

\section{Introduction}
\label{sec:intro}
1. Talk about the need for cooperative robotics here.

2. Talk about Q-Learning applied to robotics and skill acquisition here.

3. Talk about recent works that reduce the number of trials required to learn a skill.

4. Talk about recent works that remove the need for hand-coded objective functions to gain to new skills here.

5. Talk about high level contribution to solving this problem.

Empirical data has consistently proven that reinforcement learning can be used
as a viable method to learn primitive skills in a real robotic systems.
Standard Q-Learning techniques which traditionally take several hundred or
several thousand trials to converge on a learned skill have been supplanted
by techniques yielding convergent results within some small margin of error,
which require orders of magnitude fewer trials.   Although there are too many
to enumerate here, some of the techniques include the use of simple external
feedback such as "good" and "bad" indicators \cite{TAMER} or an animal
clicker \cite{Clicker}, the use of intermittent human
guidance \cite{AdviceTaking,TeacherRL}, and the integration of human
demonstration \cite{DemonstrationRL}.  These results
have made compelling the use of modified Q-Learning in real robotic systems.

Very little work has focused on using data gathered during the Q-Learning
process to later decompose a complex scene into a set of primitives via
observation.  The concept of "inverse reinforcement learning" is not entirely
novel, as Abbeel and Ng \cite{InverseRL} used inverse reinforcement learning
to generate reasonable approximations for the reward function underlying an
observed primitive.  However, their work does not leverage existing data
to create approximate classifications, but rather uses raw sensory data to
attempt to recover the unknown reward function.

We present herein a system that, using Q-Tables generated from simple
Q-Learning in a real robotic system, uses observation to 
It is important to separate the context of our contribution from that of
existing research in \textit{learning from demonstration} (LfD).  Standard
techniques in reinforcement learning have focused on using examples to
transfer a policy representing some basic skill from a teacher to a
pupil \cite{JenkinsLFD,LFDSurvey}.  We focus on the use of
\textit{existing knowledge} to generate an approximate decomposition of
complex scene data.

\section{Background}
\label{sec:background}
\subsection{Hierarchical Skill Architecture}
\cite{Hierarchical}
1. Talk about the need for hierarchical representation when multiple skills are required
2. Multiple skills makes the agent more adaptable and useful
3. Talk about decomposing skill sequences into Task plans
4. Talk about scaffolding within learning context
5. Accounting for multiple skills complicates the learning and execution problem


\subsection{Skill Recognition}
1. Talk about LBD as an accessible means for non-experts to train robots on skills.

2. Introduce concept of learning skills through traditional robot QLearner system, talk about
    result: qtable and policy indicating possible executions of skill.

3. Introduce idea that this information can also be leveraged to identify others' execution of these skills

5. Generalize result beyond gesture recognition to other motor tasks, cite konidaris' work

6. Present work as an enhancement layer for konidaris' CST work, generalizing it to automatically work for
   multiple skills, properly applying training data.

7. Present ASL recognition as an example of a complex skill set to identify, as inconsistency and noise
   are more likely prevalent with human repetitions of actions than precise robotic replay.


The domain in which we test our system is American Sign Language (ASL).
American sign language is chosen as a domain in which to test our scheme
because each motion involves six degrees of freedom -- the x, y, and z
coordinates of each hand relative to the agent's head -- which creates a
state space sufficiently large enough to preclude achieving real-time
optimality via na\"ive brute force methods.  The
robotic system has a priori knowledge of a set of ASL gestures and the
corresponding Q-Table.

Gesture recognition is not a novel concept.  Something about gesture recognition
here...\cite{HandGestures}\cite{HSMMRecognition}\cite{POMDPGesture}.
Several works have already provided robust gesture recognition systems for
ASL \cite{HoughASL}\cite{ASLRealTime}\cite{MotionASL}.

\section{Definition of Problem Space}
\label{sec:pspace}

1. Technically define the problem space being utilized, framed as the same data structure that
   can be leveraged for executing these actions. 

2. Formalize MDP being used, and each skill representation as a QLearner. Use Ng paper as a basis for this. Also include extra metadata
   being considered, such as time-to-execute and tolerance variables

3. Define policy in terms of the exploration function being utilized, explain that this function can be swapped
   depending on the agent's goals.

4. Define linear combination of policies (reward layers)

5. Define the optimal policy in terms of the training data being input

6. Define MDP state vector as being a composite of disparate sensor data -- different refresh rates
7. Self-loops permitted but do not contribute reward


\section{Primitive Recognition Algorithm}
\label{sec:recognition}
1. Re-iterate problem: Given a sample demonstration, we seek to determine which known skill is being presented (if any),
                       if it should be considered an expert demonstration, and to use the new observation to simultaneously
                       improve recognition of the action as well as improve its execution potential.

2. Unlike solving the skill execution problem, the algorithm does not have explicit control over the policy being enacted over the state space.

3. Unlike the direct inverse reinforcement learning problem, there are multiple primitive candidates and the demonstration given cannot be assumed to be an example/training data for any of them.

4. Describe Algorithm:

	*At the specified sampling rate, poll all sensors to create a state descriptor.
	*Iterate through all eligible primitive actions:
		* If transition from previous state to this state is defined and has a positive reward, add to hit_states
		  otherwise if it isn't and the sliding window, of size = antipicated_duration * tolerance * sampling_rate, contains less
		  than X percent hit_states, we time-out the primitive as an optimization step, resetting its sliding window size to 0.
		* If the current state is within an e-neighborhood of an explicitly trained goal state (don't use intuited goal states,
		  otherwise the goal area will creep outwards):
			* For each hit_state:
				* With probability p, designate current hit_state as a waypoint. Add a waypoint policy that heavily favors
				  transitions to this state from all other eligible states.
			* Compute A, the percentage of hit_states within the window.
			* Beginning with the first hit_state, compute B: the expected mean transition reward from following the waypoint policy through to a goal state. Known as the "realistic path".
			* Beginning with the first hit_state, compute C: the expected mean transition reward from following an optimal policy through to a goal state. Known as the "optimistic path".
			* Compute D, the duration elapsed when traversing the realistic path, assuming a consistent sampling rate
			* Compute E, the expected duration of an ideal example of skill execution
			* Compute F, the mean transition reward for following an optimal policy through the state space fitting the duration requirements.

			* Compute the confidence score: 0.25A + 0.4B/C + 0.1D/E + 0.25B/F
			* Apply a <score,label> tuple to each frame of data within the sliding window.



5. Analyze Algorithm:

	* Guards against failure
	* Guards against exponential state/transition growth
	* Benefits to execution side of the action primitives being trained
	* What happens with misclassification -- robustness against slippery-slope failure mode
	* Execution time required per primitive? How does the time requirement scale as primitives are added?

\section{Results}
\label{sec:result}
In this section we discuss our experimental results.

\section{Related and Future Work}
\label{sec:future}
	* Explain garbage collection to manage state space
	* Adding human feedback layer

\section{Conclusions}
\label{sec:conclusions}
In this section we conclude.

\section{Acknowledgements}
\label{sec:acknowledgements}
Get grant details from Scaz

\bibliography{report}
\bibliographystyle{aaai}

\end{document}attempt 
